%%%%%%%%%%%%%%%%%%%%%%% file template.tex %%%%%%%%%%%%%%%%%%%%%%%%%
%
% This is a general template file for the LaTeX package SVJour3
% for Springer journals.          Springer Heidelberg 2010/09/16
%
% Copy it to a new file with a new name and use it as the basis
% for your article. Delete % signs as needed.
%
% This template includes a few options for different layouts and
% content for various journals. Please consult a previous issue of
% your journal as needed.
%
%%%%%%%%%%%%%%%%%%%%%%%%%%%%%%%%%%%%%%%%%%%%%%%%%%%%%%%%%%%%%%%%%%%
%
\RequirePackage{fix-cm}
%
%\documentclass{svjour3}                     % onecolumn (standard format)
\documentclass[smallcondensed]{svjour3}     % onecolumn (ditto)
%\documentclass[smallextended]{svjour3}       % onecolumn (second format)
%\documentclass[twocolumn]{svjour3}          % twocolumn
\journalname{Fire Technology}
%\documentclass[12pt]{article}
%\setlength\topmargin{0pt}
%\addtolength\topmargin{-\headheight}
%\addtolength\topmargin{-\headsep}
%\setlength\oddsidemargin{0pt}
%\setlength\textwidth{\paperwidth}
%\addtolength\textwidth{-2in}
%\setlength\textheight{\paperheight}
%\addtolength\textheight{-2in}
%\linespread{1.25}
\usepackage{layout}
%
\smartqed  % flush right qed marks, e.g. at end of proof
%
\usepackage{graphicx}
%\usepackage{subcaption}
%\usepackage{placeins}
\newcommand{\etal}{\textit{et al}. }

\graphicspath{{figs/}}

\begin{document}

\title{Wildland Fire Spread Modeling Using Convolutional Neural Networks
}
%\author{Removed for Peer Review}
\author{Jonathan L. Hodges         \and
        Brian Y. Lattimer
}
%\institute{Removed for Peer Review \at }
\institute{J. L. Hodges \at
              Jensen Hughes\\
              2020 Kraft Drive, Suite 3020\\
              Blacksburg, VA 24060 USA \\
              Tel.: +1 540-808-2800 x10611\\
              \email{jhodges@jensenhughes.com}
}

\date{Received: date / Accepted: date}
% The correct dates will be entered by the editor


\maketitle

\begin{abstract}
This paper presents a novel predictive analytics approach to estimating
the spread of a wildland fire using a convolutional neural network
(CNN). Simulated burn maps for use in this process were
generated at six hour intervals using the phenomological fire spread model
of Rothermel with 10,000 different combinations of input parameters.
The robustness of the approach is tested using 1,000 simulations not
included when training the CNN. Overall the predicted burn maps 
from the CNN-based approach agreed with simulation results,
with mean precision, sensitivity, and F-measure of 0.97, 0.92, and 0.93,
respectively. Noise in the input parameters was found to not significantly
impact the CNN-based predictions. The computational cost of the method
was found to be comparable to a phenomological model in homogenous spatial
conditions, and significantly better for heterogenous spatial conditions.
Although trained on predictions six hours apart, the
CNN-based approach is shown to be capable of predicting burn maps
further in the future by recursively using previous predictions as inputs to
the model. When the initial fire was small, the model tended to under-predict
fire spread; however, predictions generally improved as the fire grew.

\keywords{Wildland Fire \and  Machine Learning \and Neural Network \and Fire Spread \and Convolutional Neural Network \and CNN}
% \PACS{PACS code1 \and PACS code2 \and more}
% \subclass{MSC code1 \and MSC code2 \and more}
\end{abstract}

\end{document}