%%%%%%%%%%%%%%%%%%%%%%% file template.tex %%%%%%%%%%%%%%%%%%%%%%%%%
%
% This is a general template file for the LaTeX package SVJour3
% for Springer journals.          Springer Heidelberg 2010/09/16
%
% Copy it to a new file with a new name and use it as the basis
% for your article. Delete % signs as needed.
%
% This template includes a few options for different layouts and
% content for various journals. Please consult a previous issue of
% your journal as needed.
%
%%%%%%%%%%%%%%%%%%%%%%%%%%%%%%%%%%%%%%%%%%%%%%%%%%%%%%%%%%%%%%%%%%%
%
\RequirePackage{fix-cm}
%
%\documentclass{svjour3}                     % onecolumn (standard format)
\documentclass[smallcondensed]{svjour3}     % onecolumn (ditto)
%\documentclass[smallextended]{svjour3}       % onecolumn (second format)
%\documentclass[twocolumn]{svjour3}          % twocolumn
\journalname{Fire Technology}
%
\smartqed  % flush right qed marks, e.g. at end of proof
%
\usepackage{graphicx}
%\usepackage{subcaption}
%\usepackage{placeins}
\newcommand{\etal}{\textit{et al}. }

\graphicspath{{figs/}}

\begin{document}

\title{Multiscale Fire Modeling Using Inverse Convolutional Neural Networks
}

\author{Jonathan L. Hodges         \and
        Brian Y. Lattimer
}

\institute{J. L. Hodges \at
              Jensen Hughes\\
              2020 Kraft Drive, Suite 3020\\
              Blacksburg, VA 24060 USA \\
              Tel.: +1 540-808-2800 x10611\\
              \email{jhodges@jensenhughes.com}
}

\date{Received: date / Accepted: date}


\maketitle

\begin{abstract}
This paper presents a novel predictive analytics approach to 

\keywords{Compartment Fire \and  Machine Learning \and Computational Fluid Dynamics \and CFD \and Convolutional Neural Network}

\end{abstract}




\section{Introduction}
\label{intro}

Understanding the transport of combustion products in a compartment fire is
vital in fire hazard analysis. Although advancements in computing technology
have made computational fluid dynamics (CFD) of structural fires possible,
the computational cost can be prohibitive. Parametric studies and predictions of
large structures often rely on more coarse predictions such as those from zone
fire models. Researchers have presented multiscale modeling approaches to fuse
3-D CFD with 1-D zone models; however, each approach inherits the computational
cost of CFD \cite{wang2008applications,nobile2009coupling,choi2011multiscale,colella2011novel,kuprat2013bidirectional,haghighat2018development}.
A new compartment fire model which is capable of providing rapid
predictions of 3-D spatial-temporal profile of intensive properties within a
compartment is needed.


% Paragraph about zone fire models:
% Description, advantages, disadvantages
The use of numerical tools to predict the flow of air and combustion products
in compartment fires has been an integral part of fire protection engineering
since the 1970s \cite{quintiere1976growth}. 
Zone fire models provide rapid predictions of mean values of intensive
properties in a compartment and mean transport between compartments. CFD models
provide 

% Paragraph about CFD fire models:
% Description, advantages, disadvantages
Although advancements in computing technology
have made computational fluid dynamics (CFD) predictions of structural fires
feasible, modeling large domains using CFD is beyond the capability of
current technology.

% Paragraph about Multiscale approachs:
Here I need to talk about multiscale approaches.

% Paragraph about up-convolutional neural networks
Talk about up-convolutional neural networks (chair dudes).



%Ballesteros-Tajadura et al., 2006; 
%Diego et al., 2011; 
%Stefopoulos et al., 2007;
%Yuan et al., 2007
%Choi et al., 2012; multiscale numerical analysis of airflow in CT-based subject specific breathing human lungs
%Kuprat et al., 2013; a bidirectional couplin procedure applied to multiscale respiratory modeling
%Nobile, 2009; coupling strategies for the numerical simulatino of blood flow in deformable arteries by 3d and 1d models
%Wang and Chen, 2008; applications of a coupled multizone-cfd model to calculate airflow and contaminant dispersion in built environments for emergency management
%Colella et al., 2009 calculation and design of tunnel ventilation systems use a two scale modelling approach
%Colella et al., 2011a a novel multiscale methodology for simulating tunnel ventilation flows during fires
%Colella et al., 2011b multiscale modeling of transient flows from fire and ventilation in long tunnels




The fundamental principle which makes convolutional neural networks (CNNs)
versatile is the capability to learn how to represent complex
shapes as combinations of high level feature maps. Krizhevsky showed
many of the features learned by the CNN in the ImageNet competition
described the inter-relationship of the 3 color channels \cite{krizhevsky2012imagenet}.
As an analogy to image classification, data such as elevation, moisture content, and 
wind speed can be treated as channels in an image. Given enough data, a CNN will be
able to learn relationships between these physical parameters which can then be used
to predict a future fire perimeter.

The objective of this study is to apply a CNN framework to predict the
spatial-temporal distribution of a wildland fire front in homogenous vegetation
without relying on any other models at runtime. Data for use in
training and testing the network was generated using Rothermel's phenomological
model. The sensitivity of the network to each input parameter is examined,
and the trained parameters of the network are used to infer relationships
about input parameters. The work presented herein represents a proof-of-concept
on a simple configuration with future work to expand the method to use
experimental data with heterogenous spatial conditions.


\section{Methods}
\label{s:Methods}

The method presented herein


\section{Results}
\label{s:Results}

The robustness of the 

\section{Discussion}
\label{s:Discussion}

The overall shape of 














\section{Conclusion}
\label{s:Conclusion}

A novel predictive analytics approach to 




\begin{acknowledgements}
This research was developed under Contract No. 200-2014-59669,\\
awarded by the National Institute for Occupational Safety and Health (NIOSH). The findings and
conclusions in this report are those of the authors and do not reflect the official
policies of the Department of Health and Human Services; nor does mention of trade names,
commercial practices, or organizations imply endorsement by the U.S. Government.
\end{acknowledgements}

% BibTeX users please use one of
%\bibliographystyle{spbasic}      % basic style, author-year citations
%\bibliographystyle{spmpsci}      % mathematics and physical sciences
\bibliographystyle{spphys}       % APS-like style for physics
\bibliography{structuralReferences}   % name your BibTeX data base


\end{document}

